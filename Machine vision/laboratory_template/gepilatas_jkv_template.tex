 \documentclass[paper=letter, fontsize=12pt]{article}
\usepackage[magyar]{babel} 
\usepackage{amsmath,amsfonts,amsthm} % Math packages||nem tudom kell-e
\usepackage[utf8]{inputenc}
\usepackage{blindtext}
\usepackage{graphicx}
\usepackage{amsmath}
\usepackage{physics}
\usepackage{mathtools}
\usepackage{caption}
\usepackage{subcaption}
\usepackage{verbatim}
\usepackage[sc]{mathpazo} % Use the Palatino font
\usepackage[T1]{fontenc} % Use 8-bit encoding that has 256 glyphs
\linespread{1.05} % Line spacing - Palatino needs more space between lines
\usepackage{microtype} % Slightly tweak font spacing for aesthetics
\usepackage[left=24mm,hmarginratio=1:1,top=24mm,bottom=24mm,columnsep=10pt]{geometry} % Document margins
\usepackage{url}
\usepackage{xfrac}
\usepackage{nicefrac}
\usepackage{booktabs} % Horizontal rules in tables
\usepackage{float} % Required for tables and figures in the multi-column environment - they need to be placed in specific locations with the [H] 
\usepackage{fancyhdr} % Headers and footers
\pagestyle{fancy} % All pages have headers and footers
\fancyhead{} % Blank out the default header
\fancyfoot{} % Blank out the default footer
\fancyfoot[R]{\thepage} % Custom footer text
\usepackage[nottoc]{tocbibind}
\usepackage{multirow}
\usepackage{pdfpages}
\usepackage[framed]{matlab-prettifier}
%\usepackage{titlesec}
%\titlespacing*{\subsubsection}{0pt}{2ex plus 0.2ex minus 1.5ex}{2ex minus 1.5ex}
%\titlespacing*{\section}{0pt}{1ex plus 0.2ex minus 1.5ex}{2ex minus 1.5ex}

%matlab
\let\ph\mlplaceholder % shorter macro
%\lstMakeShortInline"

\lstset{
	style              = Matlab-editor,
	basicstyle         = \mlttfamily,
	escapechar         = ",
	mlshowsectionrules = true,
}
%matlab end


\begin{document}
	
%\renewcommand{\arraystretch}{1.1}

\begin{titlepage}
	\begin{center}
		\vspace*{0.1cm}
		
		\includegraphics[width=0.65\textwidth]{mue}
		
		\vspace*{0.4cm}
		
		\huge
		\textbf{Gépi látás laboratórium jegyzőkönyv}
		
		\vspace*{2cm}		
		
		\LARGE
		\textit{n.} laboratórium

		\vspace*{0.2cm}
		\textit{Mérés időpontja}
		
		\vspace*{2cm}
		Készítette:
		
		\vspace*{0.4cm}
		\textit{Név}
		
		\vspace*{0.4cm}
		\textit{NEPTUN kód}
		
		\vspace*{2.3cm}
		
		\includegraphics[width=0.2\textwidth]{iit}
		
		\vspace*{1.5cm}		
	\end{center}
\Large
\hfill \textit{Jegyzőkönyv készítésének dátuma}

\end{titlepage}

\pagenumbering{arabic}

%\newpage
\tableofcontents
\newpage

\section{Feladatok}

\section{Megoldás (Matlab kód kommentekkel)}
A kód szövegbe illesztése miatt az összes ékezetes karaktert ékezet nélkülire kell cserélni.

\begin{lstlisting}
%% Igy hasznalhatoak Matlab kodok
% Ekezeteket sajnos nem tud kezelni a makro
I = imread('lena.png')
imshow(I)

\end{lstlisting}

\section{Eredmények}

\section{Összefoglalás, tapasztalatok}

\end{document}